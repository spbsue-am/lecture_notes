\documentclass[reqno]{article}

\usepackage{fullpage}
\usepackage{eufrak}
\usepackage{listings}
\usepackage{color}
\usepackage{xcolor}
\usepackage{enumitem}


% \usepackage{nth}
\usepackage{amsthm,amsmath,amsfonts,amssymb}


%formatting
\usepackage[left=2cm,right=2cm,top=2cm,bottom=2cm,bindingoffset=0cm]{geometry}
\renewcommand{\baselinestretch}{1.2}
\sloppy

\setlength{\parindent}{0pt}
\setlength{\parskip}{0.3em}


% Languages, fonts and symbols
\usepackage[english,russian]{babel}
\usepackage[T1,T2A]{fontenc}
\usepackage[utf8]{inputenc}

\usepackage{amsmath}

\usepackage{graphicx}

\usepackage{subcaption}

\usepackage{amssymb}

\usepackage{listings}

\usepackage{float}

\usepackage{dsfont}

\usepackage{ragged2e}

\usepackage{tabularx}


%\usepackage{algorithm}
\usepackage[ruled,vlined]{algorithm2e}

\SetKwInput{KwInput}{Input}                % Set the Input
\SetKwInput{KwOutput}{Output}              % set the Output


% tikz graphics
% Для коммутативных диаграмм.
\usepackage{tikz}
\usetikzlibrary{arrows,shapes}
\tikzstyle{vertex}=[circle,fill=black,minimum size=3pt,inner sep=0pt]
\tikzstyle{edge} = [draw,thick,-]

\usepackage{tikz-cd}

\usetikzlibrary{quotes,babel,angles}


% Теоремы и прочее
\renewcommand\qedsymbol{$\square$}

\theoremstyle{definition}
\newtheorem*{nb}{Замечание}

\theoremstyle{definition}
\newtheorem*{sol}{Решение}

\theoremstyle{definition}
\newtheorem*{exmp}{Пример}

\theoremstyle{definition}
\newtheorem*{exmps}{Примеры}

\theoremstyle{definition}
\newtheorem{exc}{Упражнение}[section]

\theoremstyle{definition}
\newtheorem{thm}{Теорема}[section]

\theoremstyle{definition}
\newtheorem*{defi}{Определение}

\theoremstyle{definition}
\newtheorem{coll}{Следствие}[section]

\theoremstyle{definition}
\newtheorem{state}{Утверждение}[section]


%Content as links
% Для того, чтобы оглавление стало ссылками, убрать комментирование кода ниже.

\usepackage[hidelinks]{hyperref}
\hypersetup{
    colorlinks,
    citecolor=black,
    filecolor=black,
    linkcolor=black,
   urlcolor=black
}


% Title
\title{Прикладные модели исследования операций.\\ Лекции.\\ ЧЕРНОВИК}
\author{Лектор: Чернов В.П.\\ Автор конспекта: Курмазов Ф.А.\thanks{f.kurmazov.b@gmail.com}}
\date{\today}


% New Symbols
\newcommand*{\divby}{\mathrel{\rotatebox{90}{$\hskip-1pt.{}.{}.$}}}%




\begin{document}

	\setlength{\parindent}{0pt}

	\maketitle

	\tableofcontents


	%%%%%
	\newpage
	\section{Лекция 1.}
		%%%
		\subsection{Модель оптимизации цен.}
			
			Дана торговая (или продуктовая) компания. Компания закупает в различных местах товары, затем привозит в некоторую точку для продажи. Себестоимость товаров различная, а также для каждого товара задан спрос в единицу времени. Требуется найти управление закупками и ценами на товары максимизирующее прибыль компании.
			
			Далее, чуть более формально, для мат. модели. Задан набор доступных продуктов $[n]$\footnote{Редко встречающееся, но красивое обозначение для первых $n$ натуральных чисел $[n] = \{1, 2, 3, ..., n\}$ без $0$.}. Для каждого вида товаров задана функция спроса $Q_k$. В частном случае, спрос представлен монотонно убывающей функцией, в еще более частном случае, который мы и рассмотрим --- линейной.

			$$Q_k = a_k\cdot p_k + b_k\ \forall\ k \in [n]$$

			Где $a \in \mathds{R}^n_-$ --- скорость убывания спроса при росте цены, $b \in \mathds{R}^n_+$ --- базовый спрос на товар при соответствующей нулевой цене, а  $p \in \mathds{R_+}^n$ --- стоимость продажи единицы товара --- переменная задачи.\footnote{Где $R_+$ --- неотрицательные вещественные числа, а $R_-$ --- неположительные.}
			
			Исходя из этого, выручку можно считать по формуле:

			$$R = \sum^n_{k=1} Q_k\cdot p_k$$	
			
			В рассматриваемой модели (простой) учитываются переменные ($V$) и постоянные ($F$) затраты, сумма которых составляет все затраты компании $C$:
			$$C = F + V$$

			Где постоянные затраты $F$ заданы (для данной простой модели), а переменные затраты $V$ рассчитываются по формуле:

			$$V = \sum^n_{k=1} Q_k\cdot v_k$$

			Где $v_k$ --- удельные переменные затраты на единицу товара.

			Требуется максимизировать прибыль $\pi \in \mathds{R}$, изменяя стоимости продажи товаров $p$:

			$$\pi = R - C \rightarrow \max$$	

			Так как введенная форма спроса допускает отрицательные значения, для математической модели задачи потребуется введение ограничений на неотрицательность спроса:

			$$Q_k \geq 0\ \forall\ k \in [n]$$		

			Дополнительно можно учитывать ограничение по затратам (бюджет) $B \in \mathds{R}$:

			$$\sum^n_{k=1}Q_k\cdot v_k \leq B$$

			\begin{nb}
				Множители Лагранжа в решении нелинейной задачи играют роль похожую на роль теневых цен в решении линейной задачи. В частности, они показывают изменение ЦФ задачи при изменении ограничения на $1$ (предельную полезность ресурса).
			\end{nb}

		%%%
		\subsection{Возможность привлечения кредитных средств.}

			Помимо прочего, в модели также можно учитывать взятие денег в кредит.			
			
			\begin{nb}
				Множитель Лагранжа при бюджетном ограничении можно интерпретировать как максимальную допустимую кредитную ставку, при взятии кредита на время операции (закупки - продажи).
			\end{nb}
			
			Введение возможности взятия кредита можно ввести с помощью добавления к бюджетному ограничению объема кредита, т.е. замены:

			$$B_{new} = B + K$$

			Где $K \in \mathds{R}_+$ --- величина кредита.

			А также с помощью добавления к издержкам процентов по кредиту:

			$$C_{new} = C + K\cdot percent$$

			Где $percent \in \mathds{R_+}$ --- процентная ставка по кредиту.


			\begin{nb}
				При решении нелинейной задачи средствами Excell стоит обратить внимание на настройку параметров (точность решения и множественные начальные точки) чтобы избежать высокой вероятности попадания в локальный минимум. При этом, похоже, что Excell использует разновидность градиентного спуска.
			\end{nb}

		%%%
		\subsection{Задача построения портфеля проектов.}

			\begin{nb}
				Это модифицированная задача о рюкзаке.
			\end{nb}
			
			Простой вариант задачи --- однопериодный. Он и описан ниже.
			
			Задано множество объектов (проектов) $[n]$, где каждый проект имеет свою стоимость $c \in \mathds{R_+}^n$ и прибыльность $\pi \in \mathds{R_+}^n$. Проект $k$ может быть взят ($y_k = 1$) или не взят ($y_k = 0$), т.е. $y \in \{0, 1\}^n$. Также задан бюджет $B \in \mathds{R_+}$ --- максимальная допустимая суммарная стоимость итогового набора проектов. Таким образом имеем задачу целочисленного программирования, в которой требуется максимизировать прибыль путем выбора наилучшего набора проектов среди всех допустимых (по бюджету), т.е. максимизировать величину:
			
			$$\sum^n_{k=1} \pi_k \cdot y_k \rightarrow \max$$

			При условии:
			
			$$\sum^n_{k=1} c_k \cdot y_k \leq B$$


	%%%%%
	\newpage
	\section{Лекция 2.}
		%%%
		\subsection{Сложный портфель.}
		
			Развитием задачи формирования портфеля (рюкзака) является расширение задачи на несколько периодов, с соответствующими изменениями во входных данных.

			Задано множество объектов (проектов) $[n]$, и множество периодов $T = \{1, 2, ..., m\}$. Для каждого периода времени $t \in T$ задан вектор доходности проектов $c^t \in \mathds{R}^n$. Также задан объем инвестиций (бюджет) в каждый период времени $B \in \mathds{R}^m$ Требуется выбрать такой набор проектов, что итоговая прибыль максимальна, и при этом ни в один период времени $t$ бюджет $B^t$ не превышается. Т.е. выполняется условие:\footnote{Обозначим $\langle \cdot, \cdot \rangle$ --- скалярное произведение.}
			
			$$\langle c^t, y\rangle + B^t \geq 0\ \forall\ t \in T$$
			
			Также возможен вариант с накоплением неиспользованных инвестиций.
			
			$$\sum\limits^{t}_{k=1} (\langle c^k, y\rangle + B^k) \geq 0\ \forall\ t \in T$$

			\begin{nb}
				В модели с переносом средств между периодами стоит учитывать дисконтирование.	
			\end{nb}
		
		%%%
		\subsection{Анализ эффективности инвестиционного портфеля.}
		
		Расчет различных показателей эффективности.
		
		\textbf{Исходные данные по проекту:}
		\begin{itemize}
			\item Даты срезов $T = \{t_1, t_2, ... t_m\}$.
			\item Поток вложений $K \in \mathds{R}^m$ на даты $T$.
			\item Поток доходов $D \in \mathds{R}^m$ на даты $T$.
			\item Ставка дисконтирования $i \in \mathds{R}^m$.
		\end{itemize}
	
		\textbf{Рассчитываемые показатели:}
		\begin{itemize}
			\item Итоговый финансовый поток: $R_t = D_t - K_t$ на даты $T$.
			\item Дисконтный множитель: $v_k = (1 + i_k)^{(- \frac{t_k - t_{k-1}}{365})}\ \forall k > 1$.
			\item Нарастающий дисконтный множитель: $V_k = \prod\limits^k_{l = 1} v_l$.
			\item Дисконтированный поток вложений: ${Kd}_k = V_k\cdot K_k$.
			\item Дисконтированный поток доходов: ${Dd}_k = V_k\cdot D_k$.
			\item Чистый дисконтированный доход $NPV_k = \sum\limits^k_{l = 1} V_l\cdot R_l = \sum\limits^k_{l = 1} (V_l\cdot D_l - V_l\cdot K_l) = \sum\limits^k_{l = 1} ({Dd}_l - {Kd}_l)$.
			\item Индекс доходности ${PI}_k = \sum\limits^k_{l=1} \frac{{Dd}_l}{{Kd}_l}$
		\end{itemize}
	
	
	%%%%%
	\newpage
	\section{Лекция 3.}
	%%%
	\subsection{Анализ эффективности инвестиционного портфеля. Продолжение}
	
	Вкладывая денежные средства в какой-то проект, мы предполагаем его окупаемость. И \emph{время окупаемости} является отличием инвестиционных проектов от любой другой коммерческой операции. Так окупаемость инвестиционных проектов происходит через длительный промежуток времени, и поэтому это время необходимо учитывать, происходит это с помощью дисконтирования. 
	
	Инвестиционный проект  оценивается через финансовые потоки (мб через другие критерии связанные со спецификой проекта). Так как мы работаем с проектом, то мы работает с прогнозными значениями. \emph{Поток вложений} (закупка оборудование, перевозка, установка...) имеет бОльшую надежность в отличие от \emph{потока доходов}(продажа продукции), так как в первом случае мы имеем информацию от поставщиков о ценах, характеристиках и др. вещах, а во втором мы не можем быть уверены, как рынок отзовется на продаваемую нами продукцию.
	
	Проводим прогноз для определения возможной прибыли/убытка.
	
	Берем ставку дисконтирования. \emph{Размер ставки дисконтирования} определяется экономической ситуации. Если экономика стабильна, то ставка относительна низкая, если нестабильна -- высокая. Чем выше ставка дисконтирования, тем хуже будут результаты по проекту. Происходит это потому, что прибыль мы сможем получить лишь спустя время. На вложениях дисконтирование скажется слабее, то есть доходы будут уменьшены сильнее, чем вложения. \\
	
	\textbf{В любом проекте присутствуют 4 базовых оценки эффективности:}
	
	\begin{enumerate}
		\item $NPV$ (чистый дисконтированный доход. Прибыль в денежных единицах)
		
		\item Индекс доходности (на столько процентов окупится наш проект)
		
		\item Внутренняя норма доходности (ставка дисконтирования при которой $NPV = 0$.)
		
		\setlength{\leftskip}{2em}
		\emph{Экономическая интерпретация}: если мы хотим финансировать проект из заемных средств, возникает вопрос - по какой кредитной ставке стоит брать кредит, чтобы была возможность расплатиться. Если кредитная ставка меньше внутренней нормы доходности, то под этот проект стоит взять кредит. Если кредитная ставка больше, то брать его не стоит.
		
		\setlength{\leftskip}{0em}
		\item Срок окупаемости проекта (разность между датой начала проекта (первый платеж по проекту) и датой когда получили в плюс (или 0) по $NPV$.)
		
		\setlength{\leftskip}{2em}
		На срок окупаемости может влиять сама ставка дисконтирования и потоки доходов. А  если к определенному моменту, нам нужна фиксированная прибыль, мы можем продать проект, тем самым получив дополнительную прибыль (изменить последний поток доходов).
	\end{enumerate}
	
	%%%%%
	\newpage
	\section{Лекция 4.}
	%%%
	\subsection{Анализ чувствительности основных показателей проекта методом Монте-Карло. Нормальный закон распределения}
	
	Нормальный закон используется, если наши потоки вложений и доходов отклоняются от средних значений из-за многих независимых внешних воздействий.\footnote{если у нас всего два-три оптовых покупателя, то нормальный закон не работает}
	
	Для анализа чувствительности дополнительно к исходным данным лекции 2 добавляются также:
	
	\begin{itemize}
		\item Случайное число
		
		\item Математическое ожидание вложений 
		
		\item Коэффициент вариации вложений ( -- сколько \% составляет стандартное отклонения от мат ожидания $\sigma / \bar{x}$. Можно также ввести дисперсию или СКО)
		
		\item Математическое ожидание доходов
		
		\item Коэффициент вариации доходов
	\end{itemize}

	В соответствии с чем изменяются потоки вложений и потоки доходов. Для этого пересчета потоков необходимо равномерное распределение перевести в нормальное. Используем функцию в Excel -- НОРМ.ОБР(вероятность(наше случ.число); среднее; стандартное отклонение). Другими словами, берем случайное число, помешаем на ось $Oy$ в закон распределения нормальной СВ и находим наше число распределенное по нормальному закону. 
	
	\textbf{Анализ чувствительности проекта}
	
	Первым этапом необходимо собрать статистику по следующим показателям:
	
	\begin{itemize}
		\item Внутренняя норма доходности $IRR$ (в Excel =ЧИСТВНДОХ(значение(наш итоговый поток); время)\footnote{для корректной работы функции, поле значение должно начинаться с отрицательного числа})
		
		\item Чистый дисконт. доход $NPV_k$
		
		\item Индекс доходности $PI_k$
		
		\item Процент случаев окупаемости проекта
		
		\item Срок окупаемости в днях (если проект окупается) 
		
		\item Число удачных испытаний
		
		\item Общее число испытаний
	\end{itemize}
	
	Далее усредняем полученные значения по всем показателям за все испытания.
	
	\textbf{Анализ чувствительности $NPV$} проводится усреднением значений $NPV$ за каждый промежуток времени.
	
	\subsection{Анализ чувствительности основных показателей проекта методом Монте-Карло. Дискретный закон распределения}
	
	Идея дискретного закона распределения состоит в том, чтобы рассмотреть конкретные варианты развития событий. 
	
	Рассмотрим вариант с пессимистической, целевой и оптимистической реализациями. У нас есть потоки вложений и доходов для каждого варианта, а также вероятности реализации того или иного варианта. Случайное число попадая в тот или иной промежуток будет характеризовать реализацию этого варианта. 
	
	Например, вероятность пессимистического варианта = 2/9; целевого = 6/9; оптимистического = 1/9. Если случайное число в интервале от 0 до 2/9, то поток вложений будет равен пессимистической величине вложений, если от 2/9 до 6/9, то целевому, если больше, то оптимистическому.
	
	
	
	
	%%%%%
	\newpage
	\section{Лекция 4.}
	%%%
	
	
\end{document}



% Examples and Shortcuts
$
	\begin{defi}
	Определение чего-то
	\end{defi}


	\begin{thm}

	\end{thm}	


	\begin{coll}

	\end{coll}	

	
	\begin{nb}

	\end{nb}	

	
	\begin{nb}

	\end{nb}	


	\begin{exmp}

	\end{exmp}	


	\begin{tikzcd}
	A \arrow[rd] \arrow[r, "\phi"] 
	&B \\
	& C \arrow[u]
	\end{tikzcd}

	% Пример мат. постановки

	$$\text{min} \quad \sum\limits_{e \in E} x_e$$
	
	$$
	\text{s.t.}
	\begin{cases}
		\sum\limits_{e \in E_v} x_e \geq 1& \forall v \in V\\
		x \in \{0, 1\}^{|E|} &
	\end{cases}
	$$

	% Пример алгоритма

	\begin{algorithm}[H]
		\SetAlgoLined
		 \SetKwFunction{FDFS}{DFS}
		 \SetKwProg{Fn}{Function}{:}{}
		 \Fn{\FDFS{$u$}}{
			$VisitFunc(u)$\;
			$visited[u] = True$\;
		        \For{$w \in \text{adj}[u]$}
						{
							\If{$visited[w] = False$}
								{
								$FoundFunc(w)$\;
								$\text{DFS}(w)$\;
								}						
							}
			$OutFunc(u)$\;
		        }

		DFS(start)
		\caption{Depth-first search}
	\end{algorithm}

