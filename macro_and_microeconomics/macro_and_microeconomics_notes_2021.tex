\documentclass[reqno]{article}

\usepackage{fullpage}
\usepackage{eufrak}
\usepackage{listings}
\usepackage{color}
\usepackage{xcolor}
\usepackage{enumitem}


% \usepackage{nth}
\usepackage{amsthm,amsmath,amsfonts,amssymb}


%formatting
\usepackage[left=2cm,right=2cm,top=2cm,bottom=2cm,bindingoffset=0cm]{geometry}
\renewcommand{\baselinestretch}{1.2}
\sloppy


% Languages, fonts and symbols
\usepackage[english,russian]{babel}
\usepackage[T1,T2A]{fontenc}
\usepackage[utf8]{inputenc}

\usepackage{amsmath}

\usepackage{graphicx}

\usepackage{subcaption}

\usepackage{amssymb}

\usepackage{listings}

\usepackage{float}

\usepackage{dsfont}

\usepackage{ragged2e}

\usepackage{tabularx}


%\usepackage{algorithm}
\usepackage[ruled,vlined]{algorithm2e}

\SetKwInput{KwInput}{Input}                % Set the Input
\SetKwInput{KwOutput}{Output}              % set the Output


% tikz graphics
% Для коммутативных диаграмм.
\usepackage{tikz}
\usetikzlibrary{arrows,shapes}
\tikzstyle{vertex}=[circle,fill=black,minimum size=3pt,inner sep=0pt]
\tikzstyle{edge} = [draw,thick,-]

\usepackage{tikz-cd}

\usetikzlibrary{quotes,babel,angles}


% Теоремы и прочее
\renewcommand\qedsymbol{$\square$}

\theoremstyle{definition}
\newtheorem*{nb}{Замечание}

\theoremstyle{definition}
\newtheorem*{sol}{Решение}

\theoremstyle{definition}
\newtheorem*{exmp}{Пример}

\theoremstyle{definition}
\newtheorem*{exmps}{Примеры}

\theoremstyle{definition}
\newtheorem{exc}{Упражнение}[section]

\theoremstyle{definition}
\newtheorem{thm}{Теорема}[section]

\theoremstyle{definition}
\newtheorem*{defi}{Определение}

\theoremstyle{definition}
\newtheorem{coll}{Следствие}[section]

\theoremstyle{definition}
\newtheorem{state}{Утверждение}[section]


%Content as links
% Для того, чтобы оглавление стало ссылками, убрать комментирование кода ниже.

%\usepackage[hidelinks]{hyperref}
%\hypersetup{
%    colorlinks,
%    citecolor=black,
%    filecolor=black,
%    linkcolor=black,
%   urlcolor=black
%}


% Title
\title{Математические модели микро- и макроэкономики.\\ Лекции.\\ ЧЕРНОВИК}
\author{Лектор: Placeholder\\ Автор конспекта: Курмазов Ф.А.\thanks{f.kurmazov.b@gmail.com}}
\date{\today}


% New Symbols
\newcommand*{\divby}{\mathrel{\rotatebox{90}{$\hskip-1pt.{}.{}.$}}}%




\begin{document}

	\setlength{\parindent}{0pt}

	\maketitle

	\tableofcontents




	\newpage
	\section{Лекция 1.}

		\subsection{Тема 1.}


	%%%%%
	\newpage
	\section{Лекция 2.}

	\subsection{Тема 1.}


\end{document}


% Examples and Shortcuts

	\begin{defi}
	Определение чего-то
	\end{defi}


	\begin{thm}

	\end{thm}	


	\begin{coll}

	\end{coll}	

	
	\begin{nb}

	\end{nb}	

	
	\begin{nb}

	\end{nb}	


	\begin{exmp}

	\end{exmp}	


	\begin{tikzcd}
	A \arrow[rd] \arrow[r, "\phi"] 
	&B \\
	& C \arrow[u]
	\end{tikzcd}

	% Пример мат. постановки

	$$\text{min} \quad \sum\limits_{e \in E} x_e$$
	
	$$
	\text{s.t.}
	\begin{cases}
		\sum\limits_{e \in E_v} x_e \geq 1& \forall v \in V\\
		x \in \{0, 1\}^{|E|} &
	\end{cases}
	$$

	% Пример алгоритма

	\begin{algorithm}[H]
		\SetAlgoLined
		 \SetKwFunction{FDFS}{DFS}
		 \SetKwProg{Fn}{Function}{:}{}
		 \Fn{\FDFS{$u$}}{
			$VisitFunc(u)$\;
			$visited[u] = True$\;
		        \For{$w \in \text{adj}[u]$}
						{
							\If{$visited[w] = False$}
								{
								$FoundFunc(w)$\;
								$\text{DFS}(w)$\;
								}						
							}
			$OutFunc(u)$\;
		        }

		DFS(start)
		\caption{Depth-first search}
	\end{algorithm}
